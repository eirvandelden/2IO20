% Beslissingen Template

\subsection*{GUI}
\subsubsection*{Nick}
\begin{itemize}
\item \textbf{Zorgverzekeraars en huisartsen:} Alle zorgverzekeraars en huisartsen zijn bekend in de
database. Er komen geen zorgverzekeraars bij en er gaan geen
zorgverzekeraars failliet. Hetzelfde geldt voor huisartsen.

\item \textbf{Afdelingen:} De namen van afdelingen worden niet gewijzigd. Er komen ook
geen afdelingen bij, er worden ook geen afdelingen opgeheven.

\item \textbf{Rekeningen:} Rekeningen (van recepten, afspraken en opnames) worden toegevoegd aan een aparte tabel
'rekening'. Recepten en afspraken worden bij het aanmaken ook direct
aan rekeningen toegevoegd. Opnames krijgen een dummyrekening-ID
toegewezen wanneer ze aangemaakt worden en ze krijgen hun
uiteindelijke ID wanneer de opname wordt be�indigd.
\end{itemize}

\subsection*{GUI: Open vragen}
\begin{itemize}
\item \textbf{Rekeningen:} De financi�le kant van de zaak levert nog een probleem op: Hoe
kan je zien of een rekening betaald is?\\
Voorstel: De zorgverzekeraar betaalt alle rekeningen voor de
pati�nt. Er komt een attribuut 'is\_betaald' bij in de tabellen
'recept', 'afspraak' en 'opname'. (vraag is opgelost)
\end{itemize}
