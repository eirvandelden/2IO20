%
%  untitled
%
%  Created by Sander on 21 december 2006.
%
\documentclass[]{article}
\usepackage[dutch]{babel}
\usepackage{geometry}
\usepackage{graphicx}
\usepackage{amssymb}
\usepackage[]{hyperref}
\usepackage{epstopdf}
\usepackage{textcomp}
\usepackage{fancyhdr}
\usepackage[official ]{eurosym}
\begin{document}

\section*{Formulieren}
\subsection{afspraakoverzichtsvenster}
In dit venster kiest de gebruiker voor een afdeling xof een dokter,
vervolgens laat het systeem het weekrooster zien met de afspraken
van die dokter of afdeling. Er kan van week naar week gebladerd
worden. Met een klik op een afspraak wordt die afspraak in een
afspraakvenster geopend. Bij een klik op een beschikbare tijd wordt
een nieuwe afspraak aangemaakt met het afspraakvenster
\subsection{afspraakvenster}
In een afspraakvenster staat een overzicht van alle gegevens van
\'e\'en afspraak met \'e\'en dokter met alle gegevens die daar
bijhoren. Vanuit dit afspraakvenster kun je terug naar het
afspraakoverzichtsvenster, en verder naar medicijnvoorschrijving.
\subsection{wachtlijstvenster}
Het wachtlijstvenster laat je een categorie afspraak selecteren,
waarop je een lijst krijgt van iedereen die voor die categorie
afspraak op de wachtlijst staat. Bij het klikken op een naam wordt
een afspraakvenster geopend om een afspraak te maken.
\subsection{case}
De gebruiker selecteert een pati\"ent en krijgt de recentste case
van die gebruiker, met alle afspraken bij die case. Er kan gebladerd
worden tussen cases.
\subsection{voorschrijving}
Dit geeft, met een case, een overzicht van de voorgeschreven
medicijnen naar aanleiding van die case. Ook kan er een nieuw
medicijn voorgeschreven worden.
\subsection{verbruik}
Na selectie van een afdeling xof een pati\"ent geeft het
verbruikvenster een overzicht van alles wat een afdeling verbruikt
heeft: zowel personeels- als materiaalkosten.
\subsection{pati\"entgegevens}
In het pati\"entgegevens kunnen patientgegevens, huisarts en
zorgverzekeraar gewijzigd worden. De pati\"entgegevens linken door
naar de cases en afspraken.
\subsection{opname}
Een overzicht van alle gegevens bij een opname. Vanuit een opname
kan doorgeklikt worden naar de case of pati\"ent die bij die opname
hoort.
\subsection{werknemer}
In het werknemersvenster staan, na selectie van een werknemer, de
persoonsgegevens, de contracten en het werkverleden van een
werknemer. Er kan worden doorgeklikt naar het rooster van die
werknemer.
\subsection{rekening}
De rekening met alle nog openstaande bedragen, na selectie van een
pati\"ent.
\end{document}
