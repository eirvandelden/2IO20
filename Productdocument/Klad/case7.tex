%
%  Usecases- case 7
%
%  Created by Xander van Heijnsbergen on 2006-12-06.
%  Copyright (c) 2006 TU/e. All rights reserved.
%
\documentclass[]{article}

% Use utf-8 encoding for foreign characters
\usepackage[utf8]{inputenc}

% Setup for fullpage use
\usepackage{fullpage}

% Uncomment some of the following if you use the features
%
% Running Headers and footers
%\usepackage{fancyheadings}

% Multipart figures
%\usepackage{subfigure}

% More symbols
%\usepackage{amsmath}
%\usepackage{amssymb}
%\usepackage{latexsym}

% Surround parts of graphics with box
\usepackage{boxedminipage}

% Package for including code in the document
\usepackage{listings}

% If you want to generate a toc for each chapter (use with book)
\usepackage{minitoc}

% This is now the recommended way for checking for PDFLaTeX:
\usepackage{ifpdf}

%\newif\ifpdf
%\ifx\pdfoutput\undefined
%\pdffalse % we are not running PDFLaTeX
%\else
%\pdfoutput=1 % we are running PDFLaTeX
%\pdftrue
%\fi

\ifpdf
\usepackage[pdftex]{graphicx}
\else
\usepackage{graphicx}
\fi
\title{Usecases - Case 7}
\author{Xander van Heijnsbergen\\}

\date{2006-12-06}

\begin{document}

\ifpdf
\DeclareGraphicsExtensions{.pdf, .jpg, .tif}
\else
\DeclareGraphicsExtensions{.eps, .jpg}
\fi

\maketitle


\begin{abstract}
\end{abstract}

\section{Usecase}

Om de medicijnverstrekking per afdeling in de gaten te houden wordt
periodiek een overzicht van de uitgeschreven medicijnen per afdeling
gemaakt. Dit wordt vergeleken met de bestellingen voor elke afdeling
en een periodieke inventarisatie van de aanwezige medicijnen.

\section{Queries}

\begin{itemize}
  \item Geef een overzicht van alle verstrekte medicijnen op een
  bepaalde afdeling in een bepaalde periode.
  \item Geef een overzicht van de medicijnbestellingen voor een
  bepaalde afdeling in een bepaalde periode.
  \item Geef een overzicht van de hoeveelheid medicijnen die op dit
  moment op een bepaalde afdeling aanwezig is.
\end{itemize}

\bibliography{}
\end{document}
