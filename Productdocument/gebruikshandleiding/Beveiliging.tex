\section{Beveiliging}\label{sec:Uitleg}
Hier wordt het een en ander uitgelegd over hoe de beveiliging van de
database in elkaar zit en hoe u deze naar uw wensen kunt aanpassen.
Elke werknemer heeft een paswoord, waarmee hij - samen met zijn
werknemers nummer, kan inloggen via het \textbf{inlog venster}.\\
Het paswoord kan door de beheerder worden ingesteld via het
beveiligings-menu of door de gebruiker zelf, nadat deze is ingelogd,
via de knop op het hoofdmenu.\\
Zodra een gebruiker is ingelogd kan hij alleen naar de schermen waar
deze recht op heeft. Deze rechten kunnen worden ingesteld in het
\textbf{beveiligins venster}. In dit venster kunt u een werknemer
selecteren en deze meer of minder rechten geven. Ook kunt u hier het
paswoord van een werknemer veranderen zonder deze werknemer te
zijn.\\
Kies een niet te kort paswoord met hoofd- en kleine letters en
cijfers!
