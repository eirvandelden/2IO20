%
%  untitled
%
%  Created by Etienne van Delden on 2007-02-08.
%  Copyright (c) 2007 __MyCompanyName__. All rights reserved.
%

\documentclass[]{article}

% Use UTF-8 encoding for foreign characters
\usepackage[utf8]{inputenc}

% Setup for full page use
\usepackage{fullpage}


% More symbols
\usepackage{amsmath}
\usepackage{amssymb}

\title{Todo Access}
\author{Etienne van Delden}

\date{2007-02-08}

\begin{document}

\maketitle

\section{Introduction}

\begin{enumerate}

  \item \checkmark Overal: sluitknoppen!
  \item \checkmark Is het mogelijk om nieuwe medicijnen toe te voegen?

  \item Hoofdvenster
    \begin{itemize}
      \item \checkmark knoppen zijn te groot
      \item \checkmark Knoppen hebben geen duidelijke namen
      \item \checkmark Caption is geen serieuze naam
      \item \checkmark Hoofd venster verdwijnt niet bij het selecteren van een menu (is dit een probleem?) (dit is geen probleem.)
    \end{itemize}

  \item Patientformulier
    \begin{itemize}
      \item \checkmark Nieuwe pati\"ent toevoegen moet makkelijker
      (subform los laten openen bij klikken op knop 'nieuwe
      patient')
      \item \checkmark Burgerlijke staat: combobox! (mogelijkheden voor
      databaseveld moeten comboboxwaarden toelaten (algemeen: alle
      waarden dus)
      \item \checkmark Geen nieuwe Patient knop, maar dat vage navigation dingetje ...
      \item Navigation dingetje ``nieuw'' is niet meteen bruikbaar (dan kan je al een nieuwe patient invullen)
      \item Nieuwe naam veld staat niet bovenaan het lijstje (??)
      \item Formulier is te lang voor het venster (en nu is het te breed)
      \item Telefoon nummer mag een ``-'' bevatten en de leading ``0'' wordt verwijdert. (het is nu een tekstveld ipv. nummer)
      \item Geslacht is geen combobox
      \item \checkmark Burgerlijke staat is geen combobox
      \item geen standaardklachten
      \item \checkmark Maak klacht hernoemen naar ``Maak nieuwe klacht aan''
      \item Tab medicijnen:
      \begin{itemize}
        \item ``naam'' moet ``medicijn'' worden.
        \item medicijnen op alfabetische volgorden sorteren
        \item Caption sub\_form .. staat er nog
      \end{itemize}
      \item Afspraken subform: Resultaat box onderaan formulier en groter maken
      \item Klacht subform: Patientveld weghalen: automatisch
      invullen!
      \item Rekening maken van een totale case (per maand)
  \end{itemize}

  \item Voorraadoverzicht:
    \begin{itemize}
      \item Formulier ontbreekt!
    \end{itemize}

  \item Werknemerformulier:
    \begin{itemize}
      \item combobox laat nog werknemer id zien.
      \item Nul valt eraf bij telefoonnummer, veld is niet breed
      genoeg.
      \item labels moeten beginnen met hoofdletter
      \item Contract sub formulier: label van salaris schaal is te groot
    \end{itemize}

  \item Gegevens toevoegen in database:
    \begin{itemize}
      \item \checkmark Soort behandeling (Eerste consult)
      \item \checkmark Soort behandeling (Controle)
      \item Opname (7 patienten die op KNO liggen waarvan er 3 op 8 mei 2007 en 4 op 10 mei 2007 worden ontslagen van de afdeling)
      \item Patient + wachtlijst KNO: Jansen A, Janssen B, Janssens C, Vries D, Vriess E, Vriesch F, Vriessen G
      \item Werknemer (Verpleegkundige A, 10 maal, afdeling KNO)
      \item Werknemer (Verpleegkundige A, HETSE B, KNO)
      \item Werknemer (Verpleegkundige A, SCHWEIZ A, KNO)
      \item Werknemer (Verpleegkundige A, ANTONIUS T, KNO)
      \item Werknemer (Verpleegkundige A, THEUNISSEN M, KNO)
      \item Werknemer (Verpleegkundige A, DONALDS, M, INTERNE GENEESKUNDE)
      \item Werknemer (Verpleegkundige A, KONING, B, INTERNE GENEESKUNDE)
      \item Werknemer (Verpleegkundige A, SNEL, Q, PLUMONOLOGIE)
      \item Werknemer (Hulppleegkundige, SOMER W, KNO)

      \item Werknemer (Hulpkracht, 10 maal, afdeling KNO of niet gespecificeerde afdeling als dat technisch mogelijk is)
      \item Werknemer (Hulpkracht, LOO J)
      \item Werknemer (Hulpkracht, RIGAT F)
      \item Soort afspraak (Controle)
      \item Patient (Crocket, D)
      \item Case (patient: Crocket, D; klacht: 'Ontstoken keelamandelen')
      \item Opname (patient: Crocket, D; van: 28 april; tot: 4 mei
      2007
      \item Werknemer (Verpleegkundige, 'Kommes, C', afdeling KNO)
      \item Werknemer (Verpleegkundige, 'Jansen, M', afdeling KNO)
      \item Werkrooster ('Jansen, M', ingeroosterd op 08-05-2007, 16:00-0:00)
      \item Werkrooster ('Jansen, M', ingeroosterd op 09-05-2007, 16:00-0:00)
      \item Werkrooster ('Kommes, C', ingeroosterd op 11-05-2007, 0:00-8:00)
      \item Werkrooster ('Kommes, C', ingeroosterd op 12-05-2007, 0:00-8:00)
      \item Ruimte (OK3, afdeling KNO)
      \item Soort afspraak ('Verwijderen buisjes')
      \item Ruimte (3, afdeling KNO)
      \item Werknemer ('MOL T', Diensthoofd afdeling Plumonologie)
      \item Patient ('BOMMEL A VAN', 'Helmond')
      \item Case ('Beknelde zenuw', 10-01-2007)
      \item Medicijn ('Ibuprofen', 10-01-2007, 31 dagen)
      \item Afspraak ('controle', 10-02-2007)
      \item Afspraak ('eerste consult', 10-01-2007)
    \end{itemize}

  \item Use cases:
    \begin{itemize}
      \item Case 1: In orde
      \item Case 2: Inroosteren personeel omschrijven
      \item Case 3: In orde
      \item Case 4: In orde
      \item Case 5: In orde
      \item Case 6: In orde
      \item Case 7: In orde (formulier mist nog)
      \item Case 8: In orde

      \item Gegevens toevoegen aan database
      \item Database klaarmaken voor gebruik op elk van de 8 Use Case startpunten
      \item Demonstratietaken verdelen
    \end{itemize}
\end{enumerate}

\end{document}
