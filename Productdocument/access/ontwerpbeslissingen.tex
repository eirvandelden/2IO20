Dit zijn een aantal ontwerpbeslissingen voor de database:
\begin{itemize}
\item Er is een nieuwe tabel rekening, deze is door een foutje 
	verbonden met de tabel patient ipv. case. Je kunt echter wel op een 
	handige manier alle dingen die nog niet verrekend zijn op een nieuwe
	rekening voor deze maand zetten en die versturen. Om ervoor te 
	zorgen dat dit goed werkt, is er in de tabellen "Afspraak", 
	"Uitgifte" en "Opname" een nieuwe veld toegevoegd om aan te geven
	op welke rekening dit staat. Dit is eventueel iets handiger dan een
   aantal aparte tabellen maken waarin wordt bijgehouden welke dingen al
   zijn verrekend. Helaas is het niet gelukt om een mooie query te maken,
   waardoor het geheel een beetje ineffici\"ent is (maar wel correct).

\item Om ervoor te zorgen dat de tabel wachtlijst een nuttige functie 
	vervuld is deze erg aangepast met iets andere velden en relaties met 
   andere tabellen. Het is nu wel mogelijk om makkelijk een patient op de
   wachtlijst te zetten en er ook op een nuttige manier weer af te halen,
   doordat er tegelijk een afspraak en een opname van wordt gemaakt. Het
   alternatief was om dit handmatig te doen, wat voor de gebruiker meer
   werk (met kans op slordigheden) betekend.

\item Het rooster is een beetje veranderd, zo is shift vervangen door een
	begintijd en een eindtijd.

\item Voor de beveiliging is er ook iets afgeweken van het uitgebreide
   model dat we in het begin hadden gemaakt (zie de takenlijst). We hebben
   ervoor gekozen om iedere werknemer een paswoord en een aantal rechten te 
   geven, en aan de hand daarvan worden er knoppen ge-en-en-disabled op het
   hoofdmenu.

\end{itemize}
