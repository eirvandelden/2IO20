%
%  Created by Nick van der Veeken on 2007-02-14.
%

\documentclass[]{article}

% Use UTF-8 encoding for foreign characters
\usepackage[utf8]{inputenc}

% Setup for full page use
\usepackage{fullpage}


% More symbols
\usepackage{amsmath}
\usepackage{amssymb}

\title{Handleiding voor degene die de demonstratie van de database aan de projectco�rdinator moet geven}
\author{Nick van der Veeken}

\date{2006-12-14}

\begin{document}

\subsection*{Scenario's (demonstratiehandleiding) - OGO Groep 2 - versie 1.007}

\subsubsection*{Use Case 1}
  \begin{enumerate}
    \item Mevrouw Flipse komt bij de dokter, zegt ze: mun nus us vurs'opt
      \begin{itemize}
        \item Voeg pati\"ent toe
          \begin{itemize}
            \item Adres: Den Dolech 2
            \item Tel nr: 040-2475118
            \item Naam: Flipse, M
            \item Geslacht: V
            \item Burg. staat: alleenstaand
            \item Zorgverzekeraar: VGZ
            \item Contactpersoon: default
         \end{itemize}

        \item Maak klacht
          \begin{itemize}
            \item Klacht: Verstopte neus
            \item Huisarts: Pierson
            \item Behandelend arts: Dendez
          \end{itemize}

        \item Maak afspraak
          \begin{itemize}
            \item Begintijd: 15:00
            \item Eindtijd: 15:15
            \item Datum: 18-04-2007
            \item Dokter: Dendez
            \item Afpraakcategorie: Eerste consult
            \item Ruimte: 1, KNO
          \end{itemize}
      \end{itemize}

    \item Mevrouw Flipse komt terug voor haar afspraak. Acties van Dendez
      \begin{itemize}
        \item Zoek afspraak van dokter Dendez op 15:00, 18-04-2007 met mevrouw Flipse
          \begin{itemize}
            \item Resultaat: krijgt inhaler voorgeschreven + operatieve neustussenschot
            nodig, wordt op de wachtlijst geplaatst
          \end{itemize}

        \item Schijf medicijn voor
          \begin{itemize}
            \item Datum: 18-04-2007
            \item Duur: 14
            \item Toediening: 2
            \item Naam: inhaler
          \end{itemize}

        \item Zet pati\"ent op wachtlijst
          \begin{itemize}
            \item Afdeling: KNO
            \item Soort afspraak: Operatie Neustussenschot
            \item Toevoegdatum: 18-04-2007
            \item Gewenste dokter: Dendez, A
          \end{itemize}
      \end{itemize}

    \item Mevrouw Flipse gaat naar de apotheek. Acties van de apotheekmedewerker
      \begin{itemize}
        \item Uitgifte medicijn
          \begin{itemize}
            \item Pati\"ent: Flipse, M
            \item Datum: 18-04-2007
          \end{itemize}
      \end{itemize}
  \end{enumerate}

\subsubsection*{Use Case 2}
  \begin{enumerate}
    \item Bekijk de bezetting van de afdeling KNO voor de week van 7 mei tot en zonder 14 mei
      \begin{itemize}
        \item Opnameoverzicht: Te ontslagen pati\"enten
          \begin{itemize}
            \item Afdeling: KNO
            \item Overzicht: Huidige ontslagen
            \item Datum: 7 mei
            \item Klik op 'zoeken'
            \item Tel de pati\"enten
            \item Datum: 8 mei
            \item Klik op 'zoeken'
            \item Tel de pati\"enten
            \item Datum: 9 mei
            \item Klik op 'zoeken'
            \item Tel de pati\"enten
            \item Datum: 10 mei
            \item Klik op 'zoeken'
            \item Tel de pati\"enten
            \item Datum: 11 mei
            \item Klik op 'zoeken'
            \item Tel de pati\"enten
            \item Datum: 12 mei
            \item Klik op 'zoeken'
            \item Tel de pati\"enten
            \item Datum: 13 mei
            \item Klik op 'zoeken'
            \item Tel de pati\"enten
          \end{itemize}
      \end{itemize}

    \item Haal pati\"enten van de wachtlijst en rooster ze in voor opname
      \begin{itemize}
        \item Ga naar het wachtlijstvenster
        \item Doe dit drie keer:
        \begin{itemize}
          \item Klik op de knop 'Pati\"ent opnemen'
          \item Neem de pati\"ent die bovenaan de wachtlijst staat
          op
          \item Begindatum opname: 09-05-2007
          \item Einddatum opname: 15-05-2007
          \item Verzin wat voor de overige gegevens
        \end{itemize}

        \item Doe dit vier keer:
        \begin{itemize}
          \item Klik op de knop 'Pati\"ent opnemen'
          \item Neem de pati\"ent die bovenaan de wachtlijst staat
          op
          \item Begindatum opname: 11-05-2007
          \item Einddatum opname: 18-05-2007
          \item Verzin wat voor de overige gegevens
        \end{itemize}
      \end{itemize}

    \item Rooster personeel in hulpverpleegkundige
    \begin{itemize}
      \item Verpleegkundige A en
      \begin{itemize}
        \item Voeg voor elke datum van 07-05-2007 tot en met
        13-05-2007 zoiets toe:
        \begin{itemize}
          \item Werknemer: HETSE B en SOMER W
          \item Datum: 07-05-2007
          \item Begintijd: 0:00
          \item Eindtijd: 8:00

          \item Datum: 07-05-2007
          \item Begintijd: 8:00
          \item Eindtijd: 16:00

          \item Datum: 07-05-2007
          \item Begintijd: 16:00
          \item Eindtijd: 0:00
        \end{itemize}
      \end{itemize}

      \item
      \begin{itemize}
        \item ..?
      \end{itemize}

      \item Vul voor 2 hulpkrachten, LOO J en RIGAT F
        \begin{itemize}
          \item Datum: 8 mei
          \item Begintijd: 8:00
          \item Eindtijd: 12:00
          \item Datum: 10 mei
          \item Begintijd: 8:00
          \item Eindtijd: 12:00
        \end{itemize}
    \end{itemize}
  \end{enumerate}

\subsubsection*{Use Case 3}
  \begin{enumerate}
    \item Maureen Jansen ruilt haar dienst op dinsdag- en
    woensdagavond met de dienst van Carolien Kommes op vrijdag- en
    zaterdagochtend:\\
    \begin{itemize}
      \item Open het werknemerscherm
      \item Ga naar werknemer Jansen, M
      \item Verander in het rooster dinsdag 08-05-2007, 16:00-0:00 in vrijdag 11-05-2007, 0:00-8:00
      \item Verander in het rooster woensdag 09-05-2007, 16:00-0:00 in zaterdag 12-05-2007, 0:00-8:00
      \item Ga naar werknemer Kommes, C
      \item Verander in het rooster vrijdag 11-05-2007, 0:00-8:00 in dinsdag 08-05-2007, 16:00-0:00
      \item Verander in het rooster zaterdag 12-05-2007, 0:00-8:00 in woensdag 09-05-2007, 16:00-0:00
    \end{itemize}
  \end{enumerate}

\subsubsection*{Use Case 4}
  \begin{enumerate}
    \item Davy Crocket moet langer blijven: \\Verander zijn geplande
    ontslagdatum van 4 mei 2007 naar 7 mei 2007.
    \begin{itemize}
      \item Ga naar pati\"ent Crocket, D
      \item Ga naar de opname van 28 april tot 4 mei 2007
      \item Verander de einddatum van 4 mei naar 7 mei 2007
    \end{itemize}

    \item Na een tijdje wachten op de wachtlijst is er is een plaats vrijgekomen voor mevrouw Flipse om geopereerd te worden: Plan een operatie in voor mevrouw Flipse op dinsdagmiddag 8 mei 2007, 14:00 uur, OK3.
      \begin{itemize}
        \item Ga naar de wachtlijst
        \item Klik op 'Pati\"ent opnemen'.
        \item Ga naar het tabblad afspraken
        \item Kies Flipse, M; Verstopte neus
        \item Vul de resterende gegevens in:
        \begin{itemize}
          \item Begintijd operatie: 14:00 uur
          \item Datum operatie: 08-05-2007
          \item Ruimte: OK3
          \item Begindatum opname: 08-05-2007
          \item Einddatum opname: 11-05-2007
          \item Bed: KNO
        \end{itemize}
        \item Klik op 'Verwerk opname'
      \end{itemize}

    \item Plan een afspraak in voor mevrouw Flipse op de KNO
    op dinsdagochtend 26 juni 2007, om 9:00 uur voor het verwijderen van de
    buisjes die bij de operatie zijn aangebracht:
    \begin{itemize}
      \item Ga naar het pati\"entscherm voor mevrouw Flipse.
      \item Ga naar afspraken
      \item Voeg een afspraak toe:
      \begin{itemize}
        \item Begintijd: 9:00 uur
        \item Eindtijd: 9:15 uur
        \item Datum: 26-06-2007
        \item Resultaat: (nog onbekend)
        \item Met dokter: Dendez, A
        \item Afspraakcategorie: Verwijderen buisjes
        \item Afdeling: KNO
        \item Ruimte: OK3
      \end{itemize}
    \end{itemize}
  \end{enumerate}

\subsubsection*{Use Case 5}
  \begin{enumerate}
    \item Mevrouw Flipse wordt uit het ZZ ontslagen: \\Hier is geen
    verdere administratie in het systeem voor nodig, omdat de
    ontslagdatum hetzelfde is als de geplande ontslagdatum.

    \item Er wordt een rekening opgemaakt: \\Maak de rekening op
    voor de casus ``behandeling neustussenschot'' d.d. 18 april 2007 van pati\"ent
    Flipse.
    \begin{itemize}
      \item Klik in het hoofdscherm op de knop 'rekeningen'
      \item Klik in het rekeningvenster op de knop 'bekijk nieuwe rekeningen'
      \item Hier kan je de rekening voor Flipse zien.
    \end{itemize}

    \item Plan een afspraak in voor controle van mevrouw Flipse op 3 juli 2007, om 9:00 uur op de afdeling KNO met dokter Dendez in kamer 3.
    \begin{itemize}
      \item Ga naar het pati\"entscherm voor mevrouw Flipse.
      \item Ga naar afspraken
      \item Voeg een afspraak toe:
      \begin{itemize}
        \item Begintijd: 9:00 uur
        \item Eindtijd: 9:15 uur
        \item Datum: 03-07-2007
        \item Resultaat: (nog onbekend)
        \item Met dokter: Dendez, A
        \item Afspraakcategorie: Verwijderen buisjes
        \item Afdeling: KNO
        \item Ruimte: OK3
      \end{itemize}
    \end{itemize}
  \end{enumerate}

\subsubsection*{Use Case 6}
  \begin{enumerate}
    \item Maak rekeningen op voor de casus ``beknelde zenuw'' d.d. 10 januari 2007 van
    mijnheer van Bommel uit Helmond. E\'en rekening gaat over januari,
    een andere over februari.
    \begin{itemize}
      \item Klik in het hoofdscherm op de knop 'rekeningen'
      \item Klik in het rekeningvenster op de knop 'bekijk nieuwe rekeningen'
      \item Hier kan je de rekening voor van Bommel voor januari en
      februari zien
    \end{itemize}
  \end{enumerate}

\subsubsection*{Use Case 7}
  \begin{enumerate}
    \item Er worden vermoedelijk medicijnen gestolen. Om dit te onderzoeken
    moet de werkelijke voorraad overeenkomen met de voorraad zoals
    opgegeven in de database.
    \begin{itemize}
      \item Ga naar het hoofdvenster en klik op 'magazijn'
      \item Klik vervolgens op de knop 'voorraadoverzicht'
      \item Iemand zal voor de vermoedelijk gestolen medicijnen
      de werkelijke voorraad moeten natellen. De aantallen horen
      overeen te komen met de aantallen die in het voorraadoverzicht
      staan
    \end{itemize}
  \end{enumerate}

\subsubsection*{Use Case 8}
  \begin{enumerate}
    \item Petra Jenssen komt bij het ZZ werken: \\Voeg een Petra toe
    als nieuwe werknemer voor 15 uur per week, flexibel, met een
    bruto maandsalaris van 2161 euro bij een 36-urige werkweek.
    \begin{itemize}
      \item Voeg nieuwe werknemer toe
      \begin{itemize}
        \item Naam: Petra Jenssen
        \item Geboorte datum: 01-08-1988
        \item Adres: Keverberg 23
        \item Tel nr: 040-293 03 68
        \item CV: C:$\backslash$DATA$\backslash$CV$\backslash$petrajenssen.doc
        \item Geslacht: V
        \item Uren: 15
        \item Sofinummer: 08069924
        \item Afdeling: Plumonologie
      \end{itemize}
      \item Voeg de CV van Petra toe aan haar werknemersgegevens.
      \begin{itemize}
        \item Contract
        \begin{itemize}
          \item Begin datum: 08-03-2007
          \item Eind datum: 08-09-2007
          \item Functie: All-round stagiair
          \item Salarisschaal: 5
        \end{itemize}

        \item Werkverleden
        \begin{itemize}
          \item Functie: Verkoop adviseuse
          \item Instituut: Apotheek Gestel
          \item Contracttermijn: 6 maanden
          \item Opleiding: Farmasci
          \item Reden ontslag: Eerste stage afgelopen
        \end{itemize}
      \end{itemize}

      \item Verdere details over Petra's aanstelling worden
      doorgegeven aan Petra's chef. Deze worden niet in de database
      bijgehouden.
    \end{itemize}
  \end{enumerate}

\end{document}
