%
%  Created by Nick van der Veeken on 2007-02-12.
%

\documentclass[]{article}

% Use UTF-8 encoding for foreign characters
\usepackage[utf8]{inputenc}

% Setup for full page use
\usepackage{fullpage}


% More symbols
\usepackage{amsmath}
\usepackage{amssymb}

\title{Scenario's oefendemonstratie}
\author{Nick van der Veeken}

\date{2006-12-14}

\begin{document}

\subsection*{Scenario's (oefensessie) - OGO Groep 2 - versie 1.007}

\subsubsection*{Use Case 1}
  \begin{enumerate}
    \item Mevrouw Flipse komt binnen in het ZZ, doorverwezen door
    huisarts Pierson voor een verstopte neus: \\Mevrouw Flipse is nog niet bekend bij het ZZ, ze wordt
    aan het pati\"entenbestand toegevoegd. Haar klacht (verstopte
    neus) wordt toegevoegd als nieuwe case, met huisarts Pierson als
    doorverwijzend arts. Dokter Dendez wordt als behandelend specialist aangesteld
    en er wordt een afspraak met Dendez gemaakt op 18 april 2007
    om 15:00 uur, op de afdeling KNO in kamer 1.

    \item Mevrouw Flipse komt op afspraak bij dokter Dendez:
    \\Mevrouw Flipse heeft een correctie van het neustussenschot
    nodig, dit moet operatief gebeuren. Ze krijgt een recept voor een inhaler voorgeschreven om de pijn te verlichten. Ook komt ze op de
    wachtlijst voor een neustussenschotoperatie.

    \item Met haar recept gaat mevrouw Flipse naar de centrale
    apotheek: \\De apotheekmedewerker pakt een inhaler uit de
    voorraad en schrijft deze uit de voorraad uit.
  \end{enumerate}

\subsubsection*{Use Case 2}
  \begin{enumerate}
    \item Kijk naar de bezetting van bedden op de afdeling KNO in de week volgend op 1
    mei 2007, 7 tot en met 13 mei: er worden op dinsdag drie pati\"enten ontslaan en
    op donderdag vier.

    \item 7 pati\"enten worden ontslagen, er moeten dus 7
    pati\"enten van de wachtlijst af worden ingeroosterd.

    \item Qua werknemers moet er op ieder moment die week een verpleegkundige A en een
    assistentverpleegkundige aanwezig zijn en op dinsdag en
    donderdag twee hulpkrachten.
  \end{enumerate}

\subsubsection*{Use Case 3}
  \begin{enumerate}
    \item Maureen Jansen ruilt haar dienst op dinsdag- en
    woensdagavond met de dienst van Carolien Kommes op vrijdag- en
    zaterdagochtend:\\
    Doe dit door achtereenvolgens Maureen Jansen vrij te geven op dinsdag- en woensdagavond 8 en 9 mei, Carolien Kommes vrij te geven op vrijdag- en
    zaterdagochtend 11 en 12 mei, Maureen Jansen in te roosteren op vrijdag- en
    zaterdagochtend 11 en 12 mei en tenslotte Carolien Kommes in te roosteren op dinsdag- en
    woensdagavond 8 en 9 mei.
  \end{enumerate}

\subsubsection*{Use Case 4}
  \begin{enumerate}
    \item Davy Crocket moet langer blijven: \\Verander zijn geplande
    ontslagdatum van 4 mei 2007 naar 7 mei 2007.

    \item Na een tijdje wachten op de wachtlijst is er is een plaats vrijgekomen voor mevrouw Flipse om geopereerd te worden:
    \begin{itemize}
      \item Plan een operatie in voor mevrouw Flipse op dinsdagmiddag 8 mei 2007, 14:00 uur, OK3.
      \item Verwijder mevrouw Flipse van de wachtlijst voor neustussenschotoperaties.
    \end{itemize}

    \item Plan een afspraak in voor mevrouw Flipse op de KNO
    op dinsdagochtend 26 juni 2007, om 9:00 uur voor het verwijderen van de
    buisjes die bij de operatie zijn aangebracht.

  \end{enumerate}

\subsubsection*{Use Case 5}
  \begin{enumerate}
    \item Mevrouw Flipse wordt uit het ZZ ontslagen: \\Schrijf
    mevrouw Flipse uit.

    \item Er wordt een rekening opgemaakt: \\Maak de rekening op
    voor de casus ``behandeling neustussenschot'' d.d. 18 april 2007 van pati\"ent
    Flipse.

    \item Plan een afspraak in voor controle van mevrouw Flipse op 3 juli 2007, om 9:00 uur op de afdeling KNO met dokter Dendez in kamer 3.
  \end{enumerate}

\subsubsection*{Use Case 6}
  \begin{enumerate}
    \item Maak rekeningen op voor de casus ``beknelde zenuw'' d.d. 10 januari 2007 van
    mijnheer van Bommel uit Helmond. E\'en rekening gaat over januari,
    een andere over februari.
  \end{enumerate}

\subsubsection*{Use Case 7}
  \begin{enumerate}
    \item Er worden vermoedelijk medicijnen gestolen. Om dit te onderzoeken
    moet de werkelijke voorraad overeenkomen met de voorraad zoals
    opgegeven in de database.
  \end{enumerate}

\subsubsection*{Use Case 8}
  \begin{enumerate}
    \item Petra Jenssen komt bij het ZZ werken: \\Voeg een Petra toe
    als nieuwe werknemer voor 15 uur per week, flexibel, met een
    bruto maandsalaris van 2161 euro bij een 36-urige werkweek.

    \item Voeg de CV van Petra toe aan haar werknemersgegevens.

    \item Voeg de gegevens over gevolgde opleidingen en arbeidsverleden toe aan Petra's gegevens.

    \item Verdere details over Petra's aanstelling worden
    doorgegeven aan Petra's chef. Deze worden niet in de database
    bijgehouden.
  \end{enumerate}

\end{document}
