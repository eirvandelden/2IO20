% Opdracht Template

% Benodigdheden voor de case:

Mijnheer Bommel uit Helmond is op 10 januari van 14:50 tot 15:25 op
afspraak geweest bij neuroloog Akkermans. Hij had last van een
beknelde zenuw en is daarvoor onderzocht in kamer 2.03. Hij is
diezelfde dag nog een keer onderzocht door Akkermans, weer in kamer
2.03. De resultaten van de
onderzoeken worden in het dossier van mijnheer Bommel bewaard.\\
Er worden een paar nieuwe afspraken gemaakt voor deze meneer, waarbij weer wordt
opgeslagen door welke dokter hij is behandeld, hoelang en wanneer. Voor alle
afspraken is tarief 2 in rekening gebracht. Alleen bij afspraak 2 zijn er
materiaalkosten van \euro 75,-. Het onderzoek is uitgevoerd tegen tarief 3.
Tarief 2 is \euro 50,- per uur, tarief 3 is \euro 75,- per uur.\\
Op het eind van elke maand krijgt meneer van Bommel een rekening waarin de kosten
van alle behandelingen van die maand verwerkt zijn.


%\subsubsection*{Interacties met de database}
%Afpsraak maken (doorgestuurd door huisarts maar dat maakt niet uit)
%\begin{itemize}
%
   %\item Soort klacht (bij afspraak)
   %\item welke kamer
   %\item welke dokter
%\end{itemize}
   %Nog een afspraak maken (maakt niet uit dat het op dezelfde dag is).\\
   %En nog een afspraak maken, maar dan op een andere dag.\\
   %
   %Ook moeten de kosten van materiaal en et tarief van een afspraak opgeslagen
     %kunnen worden (niet bij het maken van de afspraak maar als de afspraak
     %geweest is?)
   %- Rekeningen kunnen maken.
%

% queries:
\subsection*{Queries:}
\begin{enumerate}

\item Geef het rooster van de huidige dokter van de pati\"ent weer en
momenten waar mogelijk een afspraak gemaakt kan worden.
\item Geef de eerst mogelijk data voor het maken van een afspraak
bij geen voorkeur van een dokter.
\item Maak de afspraak aan in de rooster van de geselecteerd dokter.

\item Geef welke kamers er beschikbaar zijn om een afspraak te kunnen
maken.
\item Maak de afspraak voor een bepaalde kamer.
\item Geef de rekening: opzoeken hoeveel afspraken er zijn geweest in de afgelopen maand, hoeveel materiaal kosten er per afspraak waren en wat het tarief van de afspraak is geweest.
% ect.
\end{enumerate}

\subsection*{Formulieren beschrijven:}
\begin{enumerate}
\item Er moet een formulier zijn waarmee (bij de receptie ?) een afspraak
   gemaakt kan worden. Dit moet handig kunnen worden ingevuld door de
   receptionistes. Het moet mogelijk zijn om de afdeling/dokter aan te klikken
   en dan de beschikbare tijden te zien om een afspraak te kunnen maken wanneer
   er tijd is bij die dokter of op die afdeling. En om dan een pati\"ent te
   koppelen aan die tijd en zo een afspraak te maken.
\item De dokter moet kunnen invoeren hoeveel materiaal er is gebruikt bij een
   afspraak en wat voor soort afspraak het was geweest.
\item Misschien moet de dokter ook kunnen invoeren wat het resultaat van de afspraak
   was / bevindingen van onderzoek. (? hoort bij het vorige punt) 
\item Een formulier om aan het einde van de maand een rekeningen te kunnen maken
   voor alle afspraken die er die maand zijn geweest (per persoon) en dan
   eventueel ook alle nog niet betaalde afspraken. Per afspraak wordt gekeken
   welke materialen er zijn gebruikt en wat voor soort afspraak het was geweest,
   om aan de hand daarvan het standaard tarief te bepalen voor deze afspraak.
\end{enumerate}
