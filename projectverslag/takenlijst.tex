% takenlijst van OGO 1.2 groep 2. Jaargang '06-'07, trimester 2.
%%% Nick, 07-12-2006, geschikt gemaakt voor include in productdocument.

%%%\documentclass[11pt]{report}

%% Gebruikte pakketten
%%%\usepackage[dutch]{babel}
%%%\usepackage{geometry}
%%%\usepackage{graphicx}
%%%\usepackage{amssymb}

%% Het Document
%%%\begin{document}

% leg situaties uit en formuleer queries
%%%\chapter*{takenlijst}
Elk personeelslid heeft schrijftoegang tot pati\"entgegevens, zijn
eigen rooster en afspraken en de voorraad medicijnen.
\begin{itemize}
\item Geneesheer-directeur (GD):\\
    De GD heeft leestoegang tot alles.
\item Personeelsfunctionaris (PF):\\
    De PF heeft volledige toegang tot de personeelsdatabase.
\item Afdelingshoofd (AH):\\
    De AH's hebben leestoegang tot alle agenda's, volledige toegang
    tot de roosters en afspraken van zijn afdeling, tot de patientendossiers van zijn afdeling en leestoegang tot de
    personeelsgegevens, voorraden en bedden.
\item Diensthoofd (DH):\\
    De DH's hebben dezelfde toegangsrechten als de AH's.
\item Consulenten (C):\\
    C's hebben volledige toegang tot pati\"entendossiers en zijn
    eigen afspraken.
\item Paramedisch personeel (PP):\\
    PP's hebben toevoegrechten aan de pati\"entendossiers.
\item Baliemedewerkers (BM):\\
    BM's hebben volledige toegang tot pati\"entgegevens en afspraken.
    Ze hebben leestoegang tot opnamegegevens.
\item Verpleegkundigen (VK):\\
    VK's hebben leestoegang tot pati\"entendossiers.
\end{itemize}

%%%\end{document}
