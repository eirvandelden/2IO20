% Beslissingen Template

\subsection*{Presentatie}
Er is er voor gekozen om te werken met het programma ``Keynote'' in de plaats van PowerPoint. Keynote biedt geavanceerdere functionaliteiten voor het eenvoudig opstellen van presentatie met onder andere transities die Powerpoint niet bied. Ook hebben wij nu gebruik kunnen maken van een apart scherm voor de presentator, tijdens de presentatie. \\
Wij hebben gekozen voor een klant gerichte presentatie. Wij hadden ook kunnen kiezen voor een collega gerichte presentatie, maar wij wouden graag oefenen met het presenteren naar de klant, dit is namelijk wat wij, voor sommige van ons, uiteindelijk na de studie moeten doen.\\
Wij hebben wel demonstraties gegeven, om de presentatie interactiever te maken.


\subsection*{Beveiliging}
Er is een beveiliging ge-implementeerd. Deze werkt op een redelijk simpele
manier, waarmee elke gebruiker een inlogcode (werknemer nummer) en een
paswoord krijgt en aan de hand daarvan bepaalde dingen met de database wel
of niet kan doen. Dit is een vrij simpel beveiligingssysteem en zou
natuurlijk niet afdoende voor een serieus project, maar voor een
ogo project leek het ons afdoende.\\
We hebben ervoor gekozen om aan de hand van de rechten die een gebruiker
heeft bepaalde knoppen op het hoofdformulier aan en uit te zetten en zo
toegang tot bepaalde handelingen met de database te toe- en ontzeggen.\\
Deze rechten zijn gebaseerd op de takenlijst en wat er sowieso met de
database kan (en moet). Voor een arts is het dus grotendeels alleen nodig
om patient-informatie en afspraken te zien, terwijl het voor mensen van de
financi\"ele administratie ook nodig moet zijn om rekeningen en informatie
over het magazijn te bekijken. De formulieren die voor deze dingen nodig
zijn kunnen met rechten worden aangepast aan de gebruiker.


\subsection*{Projectverslag}
We hebben besloten het projectverslag op te delen in twee stukken:
het productdeel en het procesdeel. Dit hebben we gedaan omdat bij
ogo 1.1 expliciet ge\"eist werd. Bovendien was de hoeveelheid werk
zo te verdelen over twee personen.
