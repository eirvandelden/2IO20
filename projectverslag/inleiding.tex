Het Zorgvliet Ziekenhuis wil een informatiesysteem dat al haar
informatie koppelt en inzichtelijk maakt. Het huidige systeem
voldoet niet en er is besloten een nieuw systeem te laten maken, om
de downtime tot
een minimum te beperken.\\
Dit informatiesysteem moest voor onze ogo-opdracht, door ons worden
ontwikkeld.\\
De opdracht die door de directie van het weliswaar fictieve
Zorgvliet Ziekenhuis van grote waarde wordt geacht staat uitgebreid
gespecificeerd in de casus.\\
Dit verslag poogt een overzicht te bieden van de werkwijzen en
productontwikkelingen bij de totstandkoming van dit
informatiesysteem.\\
\\
\begin{itemize}
    \item Ontwerp\\
    In het onderdeel ``Ontwerp'' worden de Use Cases, de fictieve
    handelingen die dagelijks in het ziekenhuis voor zouden kunnen
    komen, beschreven en de bijbehorende formulieren. Tevens worden
    besluiten en aannames besproken.
    \item Modellen\\
    In het onderdeel ``Modellen'' wordt de totstandkoming van het
    ER-model besproken.
    \item Structuurbeschrijving\\
    In het onderdeel ``Structuurbeschrijving'' staan de handleiding
    van de database en drie mogelijke verbeteringen aan de database.
    \item Reflectie\\
    In het onderdel ``Reflectie'' evalueren wij elkaar en de
    opdrachten.
\end{itemize}
