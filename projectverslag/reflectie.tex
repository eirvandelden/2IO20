\subsection{Giso Dal}

Er is veel veranderd in OGO 1.2 ten opzichten van OGO 1.1. De klein
hakkelingen die ervoor zorgden dat het allemaal niet zo soepel
verliep zijn vrijwel weg. Afspraken zijn duidelijker en de
taakverdeling eveneens ook. OGO 1.2 heeft wat dat betreft meer
inzicht geleverd over hoe je moet samenwerken in de groep. Nu
blijven er toch wel punten over voor verbetering die zeker mee
genomen zullen worden in het volgende OGO project. De communicatie
onderling is een punt waar altijd aan gewerkt kan worden. Ook al was
er een taakverdeling viel grotendeels van het werk bij de
verantwoordelijk, wat niet de bedoeling was. Iedereen hoort aan een
bepaalde taak mee te werken. Zo heeft jurjen merendeel gewerkt aan
aan de database. Ik zelf en Sander hebben heel het project verslag
in elkaar gedraaid en Etienne en Nick hebben de demonstratie en
presentatie verzorgd. Persoonlijk zou ik het fijn hebben gevonden om
ook in andere onderdelen een groter aandeel te hebben gehad. Maar
mede door tijdsdruk is dat er niet van gekomen. Al met al is er veel
geleerd en ben ik heel tevreden met het eindproduct.

\subsection{Etienne van Delden}

Je kunt nooit genoeg leren, hoe oud je ook bent. Zo heb ik tijdens
OGO 1.1 al meer geleerd over samenwerken, iets waar ik nu mee verder
verrijkt ben. OGO1.2 is qua opbouw sowieso anders, omdat je nu aan
een fysieker project werkt, een programma waar je mee kan werken,
waar je dingen kan invullen en om kunt klagen.
\\\\
Met 2 projectleden had ik al vaker gewerkt, Giso Dal en Nick van der
Veeken. Giso gaf bij mij dezelfde indruk als bij OGO1.1, hij is een
hardewerker, maar toont weinig tot geen initiatief. Ik kan echter
niet klagen, als je zegt wat hij moet doen, dan doet hij doet en
lever kwaliteit af voor of op de deadline. Nick is ook een
harderwerker, maar af en toe tot in den extreme en doet alles op
zijn eigen manier. Als meehelpt met het project van een ander, dan
doet hij alles op zijn manier, ipv zich aan te passen op de
originele opbouw. Sander en Jurjen waren nieuw voor mij. Jurjen viel
vooral op. Echter niet op positieve wijze. Hij heeft een instelling
van "als ik een 6 haal dan is het goed", toont geen initiatief, doet
niet een kwaliteitschecks en zegt af en toe voluit dat ie ergens
geen zin in heeft en het ook werkelijk niet doet. Ook mist hij het
idee dat dit een project voor een klant moet voorstellen, het is een
voorbereiding op een situatie die later in zijn leven zal voorkomen.
Sander was een goede werker. Hij is goed in het motiveren van
anderen, werkt hard en toont initiatief. Hij ziet in wat de
kwaliteiten en specialiteiten van anderen zijn en speelt daar op in.
Echter heeft hij af en toe misschien de neiging om teveel hooi op de
vork te nemen.

\subsection{Sander Leemans}

Mensen veranderen. Zo ook studenten: waar er bij ogo 1.1 nog sprake
was van flinke communicatiestoornissen, de daaruit volgende
misverstanden en de zo nu en dan de kop opstekende luiheid, was
iedereen er deze ogo, 1.2, gelukkig van doordrongen dat de
hoeveelheid werk dat lonkend op ons wachtte een dergelijk soort
dingen niet zou toelaten. Naast het uiten van peptalk bestaat een
reflectie natuurlijk ook uit een paar kritische noten. Zo zou Nick,
onze artistiekeling, zich wat meer mogen focussen op hetgeen hij aan
het doen is, zo bleek ook tijdens de eindpresentatie: alles was tot
in de puntjes voorbereid, precies zoals bij de demonstratie, maar
hij week af van het gebaande pad met een foutmelding tot gevolg.
Etienne neigt soms, evenals Nick, te snel afgeleid te worden, maar
heeft, in tegenstelling tot Nick, dit dan zelf in de gaten. Etienne
doet verder ook echt zijn best. Giso is het hardstwerkende lid van
ons team, maar het initiatief ontbreekt hem. Jurjen ontbreekt
initiatief en nauwgezetheid, wat samenwerken met hem soms ernstig
bemoeilijkt. Ikzelf had mij bij het vorige ogo-project voorgenomen
om me wat meer op de achtergrond te houden. Dit is me, vooral in het
begin, goed gelukt. Het is me echter niet goed bevallen, omdat
niemand de taak op zich nam om het geheel aan te sturen, waardoor
het project soms dreigde te verzanden en bepaalde eisen uit de
projectwijzer
simpelweg bij niemand bekend waren.\\
Over de opdracht ben ik positief, want hij was nauw genoeg om
duidelijk te hebben wat er van ons verlangd werd, en ruim genoeg om
daar onze eigen invulling aan te geven.


\subsection{Jurjen Middendorp}

Het werken met access was in het begin een beetje frusterend, omdat
je voor alle opties erg veel moest zoeken totdat je had gevonden wat
je nodig had. Op het eind ging dit iets beter, maar het blijft een
irritante manier om dingen te maken. Ook was het jammer dat de
uitleg bij de access trainingen niet echt aanwezig was en het
document af en toe dingen op een ietwat lullige manier aanpakte,
waardoor het allemaal erg onoverzichtelijk
overkwam en niet iedereen er veel zin in had.\\
Het was ook goed te merken dat we nog niet zoveel ervaring hadden
met het maken van iets grotere projecten, zowel bij de planning als
bij het maken van voorbereidend werk als bij het op tijd afhebben
van dingen is er af en toe niet al te precies gewerkt, waardoor we
soms een beetje in tijdnood kwamen of er soms dingen nog niet waren.
Uiteindelijk is alles toch nog wel
goed gekomen en hebben we alles op tijd afgekregen.\\
Ook bij het bijhouden van logboeken en documenteren van wat we
hadden
gedaan was dit te merken.\\
Het groepsproces ging op zich wel goed, we hebben veel en lang
vergaderd
over dingen, en er zijn meestal goede notulen van gemaakt.\\
Het was jammer dat Xander halverwege het project stopte met de
studie, we hebben geprobeerd om het werk dat hij heeft gedaan nog
een beetje te verbeteren, maar uiteindelijk is het gedeelte wat hij
zou doen toch een
beetje onaf gebleven.\\

\subsection{Nick van der Veeken}

OGO-projecten hebben iets speciaals. Het is een vorm van onderwijs
die doet denken aan samenwerken in teams zoals in de bedrijfswereld
en tegelijkertijd valt op dat het oefenen van academische
vaardigheden hiermee gecombineerd wordt. OGO een voorbereiding op
het echte leven in de grote boze buitenwereld, waarbij de oefening
enerzijds realistisch is en anderzijds zonder risco, aangezien er
geen echte klant met overvulbare wensen aan te pas komt.

Nu de beoordeling over de groepsleden, in de stijl van het al uit
OGO 1.1 bekende 'peer-review': Etienne is een enthousiaste leerling
die ook al zichtbaar ervaring heeft met presenteren en verkopen. Hij
heeft een duidelijke en onderbouwde mening die ik graag hoor tijdens
vergaderingen. Sander houdt van een efficiente werkwijze: geen
onnodige besprekingen, woorden vuilmaken is onnodig, dus zwijg en
werk door. Concentratie is zijn sterke punt. Jurjen is
verbazingwekkend weerbarstig: hij heeft als eerste reactie: ``maar
dat is helemaal niet nodig''. Mijn visie op efficient is anders,
kwaliteit afleveren, ook als het een paar minuten extra kost.
Wanneer je hem bezig laat in zijn eigen wereldje kan hij wel aardige
formulieren maken, maar opmaak is dan blijkbaar weer niet nodig.
Utilistisch, maar vanuit een niet allesomvattend wereldbeeld. Wat
Giso allemaal doet is mij een raadsel. Hij zit achter zijn laptop,
zegt iets terug als je hem wat vraagt, maar of hij ook een mening
heeft is mij onduidelijk gebleven. Uitvoeren van taken gaat
redelijk, maar initiatief moet je niet van hem verwacht. Qua
verantwoordelijkheidsgevoel wil ik graag Etienne en Sander een pluim
geven. Ook al ik een keer wat vergeet wordt ik hierop gewezen. Zelf
had ik soms wat serieuzer uit de hoek kunnen komen, maar
verantwoordelijk ben ik zeker geweest. Jammergenoeg was er geen
voorzittersfunctie, waardoor ik met mijn idee\"en over het reilen en
zeilen niet goed weg kon. Als er een voorzittersfunctie was had ik
deze graag op me genomen; nu voel ik me enigzins medeplichtig aan
het niet altijd soepel verlopen van de samenwerking.
