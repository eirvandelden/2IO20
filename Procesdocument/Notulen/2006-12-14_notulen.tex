%
%  Created by Nick van der Veeken on 2006-12-13.
%
\documentclass[]{article}

% Use utf-8 encoding for foreign characters
\usepackage[utf8]{inputenc}

% Setup for full page use
\usepackage{fullpage}

% Uncomment some of the following if you use the features
%
% Running Headers and footers
%\usepackage{fancyheadings}

% Multipart figures
%\usepackage{subfigure}

% More symbols
\usepackage{amsmath}
\usepackage{amssymb}
%\usepackage{latexsym}

% Surround parts of graphics with box
\usepackage{boxedminipage}

% Package for including code in the document
\usepackage{listings}

% If you want to generate a toc for each chapter (use with book)
\usepackage{minitoc}

% This is now the recommended way for checking for PDFLaTeX:
\usepackage{ifpdf}

%\newif\ifpdf
%\ifx\pdfoutput\undefined
%\pdffalse % we are not running PDFLaTeX
%\else
%\pdfoutput=1 % we are running PDFLaTeX
%\pdftrue
%\fi

\ifpdf
\usepackage[pdftex]{graphicx}
\else
\usepackage{graphicx}
\fi
\title{Notulen}
\author{Nick van der Veeken\\ }

\date{2006-12-14}

\begin{document}

\ifpdf \DeclareGraphicsExtensions{.pdf, .jpg, .tif} \else
\DeclareGraphicsExtensions{.eps, .jpg} \fi

\maketitle

\section{Opening}
  Voorzitter: Giso\\
  Notulist: Nick\\
  Aanwezigheid: Iedereen is aanwezig.\\
  \\
  Hier moet volgende keer eerst door de voorzitter worden gevraagd
  of iedereen het eens is met de punten die op de agenda staan.
  Dit is niet gebeurd bij deze vergadering.

\section{Notulen vorige vergadering}
  De vragen m.b.t. de inhoud van de takenlijst zijn inmiddels
  beantwoord door de expert. Nick en Sander hebben een takenlijst
  gemaakt en op de SVN-server geplaatst ter beoordeling door de
  groepsleden.\\
  \\
  Er staat nog geen link naar de SNV-server in de kladmap op
  Studyweb. Dit is een taakje voor de SVN-serverbeheerder, Xander
  dus.\\
  \\
  De persoon die verantwoordelijk is voor het procesdocument is per
  definitie ook verantwoordelijk voor het verslag dat gemaakt moet
  worden over hoe de vergaderingen invloed hebben gehad op het
  ontwerp en de productie van het eindproduct. Dit is dus Sander.
  Nick zal het werkplan hier naar aanpassen.\\
  \\
  Wat betreft de vergadertraining heeft de tutor nog een aantal
  opmerkingen. Het is niet erg om zaken m.b.t. het groepsproces meer
  dan duidelijk vast te leggen. Dit zorgt ervoor dat er tussen leden
  onderling onderbouwd kritiek kan worden gegeven die vervolgens ook
  tot verbeteringen kan leiden. Een duidelijk werkplan bijvoorbeeld,
  zorgt ervoor dat je weet waar je aan toe bent. Afspraken die niet
  direct met het product of met deadlines te maken hebben kunnen in
  een (aantal) alinea's onderaan het werkplan worden opgenomen.\\
  \\
  Nick heeft gisteren het studentenoverleg bezocht en hiervan
  notulen gemaakt. Deze staan op Studyweb en op SVN. De persoon die
  naar het volgende overleg gaat moet nog worden aangewezen. Nick
  zal hiervoor rondvragen en daarna dit aangeven in het werkplan.\\
  \\
  Het procesdocument (vernoemd als projectverslag in het werkplan,
  wordt gecorrigeerd door Nick) moet een reflectie bevatten van de
  groepsleden op het verloop van het project, zoals ``Wat heb ik van dit project
  opgestoken?'' en ``Welke zaken verdienen bij een volgend project
  meer aandacht?''. Het procesdocument hoeft geen opzichzelfstaand
  document te zijn, het mag, in de vorm van enkele hoofdstukken,
  deel zijn van het productdocument. Agenda's en notulen kunnen
  hierbij worden opgenomen, maar dit is niet noodzakelijk, aangezien
  ze vooral bedoeld zijn voor eigen gebruik binnen de groep.\\
  \\
  Als laatste puntje: Het bespreken van de notulen van de vorige
  vergadering houdt niet in dat de inhoud ervan wordt besproken. Het
  is de bedoeling dat er gekeken wordt naar spelfouten of mogelijke
  onduidelijkheden en naar de volledigheid. De bovengenoemde punten
  horen dus eigenlijk elders in de vergadering thuis.

\section{Mededelingen tutor}
  Uiteindelijk komen de medelingen van de tutor aan bod, voor zover
  hij zaken nog niet bij de bespreking van de notulen van de vorige
  vergadering had genoemd. Het gaat over de ingeleverde opdrachten van de
  Access-training en de KZH-opdracht. Onvoldoendes en nog niet
  ingeleverde opdrachten moeten door de leden zelf zo snel mogelijk
  in orde worden gemaakt door ze (opnieuw) in te leveren. De
  beoordeling van de KZH-opdracht volgt volgende week. De resultaten
  zijn als volgt:\\
  \\
  \begin{itemize}
    \item Giso: Onvoldoende voor H. 1 t/m 4 van de Access-training.
    Bepaalde knoppen werken niet goed. De sluitknop sluit niet heel
    Access af, maar slechts de database.
    \item Etienne: Zelfde verhaal als bij Giso, echter was de
    KZH-opdracht nog niet ingeleverd. Etienne geeft aan dat dit
    gisteren al gebeurd is.
    \item Xander: Goed.
    \item Jurjen: Onvoldoende, het startformulier wordt niet
    geopend.
    \item Sander: Goed.
    \item Nick: Voldoende. Sommige knoppen deden het niet, maar de
    rest was in orde.
  \end{itemize}

\section{Mededelingen leden}
  De logboeken waren niet allemaal op tijd bijgewerkt,
  waardoor het niet mogelijk was het logboek op tijd samen te
  voegen. Afgesproken is dat de logboeken voortaan iedere week op dinsdag om
  23:59 bijgewerkt moeten zijn. Nick zal vervolgens de
  samengevoegde versie voor 17:00 op woensdag op Studyweb
  plaatsen.

\section{Bespreking opdracht}
  De deadline voor het ER-model zal gehaald worden. Er is een paar
  keer met de heer Voorhoeve gemaild, hij meldde dat we goed bezig
  zijn. Vanmiddag wordt er verder gewerkt aan het ER-model.

\section{Verslaglegging}
  Dit punt was al besproken in de voorgaande punten. Volgende punt.

\section{Planning}
  Prototype GUI: Sander vraagt aan de tutor of er naast een tekstuele
  beschrijving van de GUI ook nog daadwerkelijk formulieren gemaakt
  moeten worden in Access. Dit wordt door de tutor gezien als ``veel
  werk dat eigenlijk aan het eind van het project moet worden gedaan en dat veel tijd in beslag zal
  nemen''. Marc Voorhoeve zal onze vragen omtrent de GUI kunnen
  beantwoorden. Er wordt een mail hierover naar de heer Voorhoeve
  verstuurd.\\
  \\
  De deadlines zoals genoemd in het werkplan zijn harde deadlines.
  Giso stelt voor om interne deadlines te stellen die voor de harde
  deadlines liggen. Voorlopig wordt steeds 23:59 op de woensdag voor
  de harde deadline aangehouden als interne deadline. Nick zal dit
  in het werkplan aanpassen.
  \\
  De tutor heeft ook nog een opmerking over deadlines: Het is leuk
  om te weten wanneer een taak af moet zijn, maar wanneer begin je
  er aan? Welke andere taken moeten afgewerkt zijn voordat je aan een volgende taak kan beginnen? Wanneer
  m\'oet je aan een taak beginnen, wil je dat deze af is voor de
  gestelde deadline?\\
  Hierover wordt komende week door de groepsleden nagedacht. Dit is
  vooral belangrijk voor het laatste deel van het project, omdat er
  dan al verkennend werk kan worden gedaan om problemen in de
  planning vroegtijdig aan het licht te laten komen en indien nodig
  de planning aan te passen.\\
  \\
  De tutor merkt alvast op dat op 9 februari hij afwezig zal zijn,
  wat tot gevolg heeft dat de deadline die voor die week staat waarschijnlijk wat
  vervroegd zal moeten worden.

\section{Evaluatie}
  De volgorde waarin de punten op de agenda zijn besproken was niet
  zoals de bedoeling is. Sander is volgende week voorzitter, hij zal
  hier op letten bij de volgende vergadering.\\
  \\
  Het ER-model wordt vandaag afgewerkt en ingeleverd. Er wordt ook
  verder gewerkt aan de tekstuele omschrijving van de GUI en mocht
  het nodig zijn dan wordt die volgende week verbeterd en eventueel
  uitgebreid met een meer tastbare vorm van prototype.

\section{Sluiting}
  Giso sluit de vergadering om 14:40.\\
  \\
  De volgende vergadering vindt plaats op 21 december, tijdens de
  vergadertraining in Traverse 5, om 14:30. Sander is dan voorzitter en Xanders is
  notulist.

$\boxtimes$

\end{document}
