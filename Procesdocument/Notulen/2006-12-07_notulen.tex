%
%  Created by Jurjen on 7-12
%
\documentclass[]{article}

% Use utf-8 encoding for foreign characters
\usepackage[utf8]{inputenc}

% Setup for fullpage use
\usepackage{fullpage}

% Uncomment some of the following if you use the features
%
% Running Headers and footers
%\usepackage{fancyheadings}

% Multipart figures
%\usepackage{subfigure}

% More symbols
\usepackage{amsmath}
\usepackage{amssymb}
%\usepackage{latexsym}

% Surround parts of graphics with box
\usepackage{boxedminipage}

% Package for including code in the document
\usepackage{listings}

% If you want to generate a toc for each chapter (use with book)
\usepackage{minitoc}

% This is now the recommended way for checking for PDFLaTeX:
\usepackage{ifpdf}

%\newif\ifpdf
%\ifx\pdfoutput\undefined
%\pdffalse % we are not running PDFLaTeX
%\else
%\pdfoutput=1 % we are running PDFLaTeX
%\pdftrue
%\fi

\ifpdf
\usepackage[pdftex]{graphicx}
\else
\usepackage{graphicx}
\fi
\title{Notulen 07-12-2006}
\author{groep 1b (of 2) }
%Voorzitter: Xander
%Notulist: Jurjen
\date{07-12-2006}

\begin{document}
\maketitle
\section{Bespreken notulen vorige vergadering}
Er was een zin uit de vorige notulen niet duidelijk. Volgens de tutor had
hijzelf nooit het woord ``rest'' gebruikt om daarmee proces en product documenten
mee aan te duiden. Er wordt ter verduidelijking nog een keer afgesproken dat
notulen en dergelijke op studyweb komen en dat de rest (bijvoorbeeld
proef-product-versies) via svn ingeleverd kunnen worden.

\section{Mededelingen tutor}
De tutor vraagt of er in deze groep personen waren geweest die niet bij de
Visio-training zijn geweest. De schuldige (Jurjen) meldt zich en zegt toe om de
inhaal-opdracht te maken.\\
Er stonden ook nog geen logboeken op studyweb. Dit komt zo snel mogelijk.\\
De notulen van de vorige vergadering stonden in de verkeerde map op studyweb. Ze
moeten in de Proces map staan (of een submap daarvan) maar niet in de klad
map.\\
Als het nodig is om aanpassingen aan de notulen te maken moet dat zowel in de
notulen van de vergadering waarin dit is besloten staan als in die notulen
zelf.\\
De zalen en plekken die in de studiegids staan om ogo te doen zijn niet goed. De
juiste zalen zijn: HG 8.58, HG 10.56 en HG 10.62.\\

%sorry mensen, op dit moment was de vergadering al een hele tijd bezig, en het
%was eerlijk gezegd nogal saai, mijn excuses dat de kwaliteit van mijn notulen
%daar een beetje onder geleden heeft.

\section{Mededelingen groepsleden}
Etienne had een vraag over de taakverdeling van gebruikers van de
database. Nick had een vraag over de taakoverzicht lijst van de
medewerkers van het ziekenhuis en of dit gerelateerd moest zijn aan
de use cases. In beide gevallen kwam het antwoord van de tutor er
ongeveer op neer dat we de opdracht goed moesten lezen en daarna
met inhoudelijke vragen naar de expert moeten gaan en niet naar de
tutor.\\

\section{Over de opdracht van deze week}
We moeten na de vergadering nog wel even aan de opdrachten voor vrijdag werken,
maar we denken dat het wel af gaat komen.\\
\section{Random vergader punten}
Er is een uitgebreide discussie over de definitie van presentatie en
demonstratie. Uiteindelijk komen we op het volgende uit:\\
\begin{itemize}
\item Presentatie: Een terugblik op het groepswerk, en een statische uitleg van
het gedane werk (screenshots).
\item Demonstratie: Een volledig interactieve real-time demonstratie van
een product.
\end{itemize}
Gelukkig krijgen we nog een presentatie training waarin de subtielere
verschillen tussen deze twee uitgebreid aan de orde zullen komen. %yay
Eventuele vragen zullen ook aan Sidorova gesteld kunnen worden tijdens het
sudentoverleg.\\
Voor meer informatie over planning en het procesdocument kunnen we ook kijken op
de OGO-internet pagina. Onder andere is er daar een link naar Atos (?). Bij die
presentatie staat rond slide 30 interessante dingen over het bewaken van
planning en proces.\\
De hoofdverantwoordelijke voor "Bespreking vergaderingen en gevolgen ontwerp en
realisatie" moet een groepslid zijn.\\
Ons schema voor deadlines was niet goed. Deadlines voor het werk van groepsleden
moeten eerder worden gesteld dan de deadlines waarop we het werk uiteindelijk
moeten inleveren. Dit om eventuele fouten en misverstanden en daardoor
vertragingen te voorkomen. 

De volgende vergadering is volgende week (14-december 2006) om 14:00 uur in
H8.58.\\
De voorzitter sluit zo rond 15:15 de vergadering.\\

De vergaderjuffrouw had nog een opmerking over dat we erg lang vergaderen en
daarom niet in staat zijn geweest om nog een extra vergadering te houden over
``groepscriteria en groepsregels''. Als we dat leuk vinden mogen daar in onze
vrije tijd nog een keer over vergaderen.

\end{document}
