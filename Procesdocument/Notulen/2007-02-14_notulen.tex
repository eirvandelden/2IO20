%
%  Created by Etienne on 14 februari 2007.
%
\documentclass[]{article}

% Use utf-8 encoding for foreign characters
\usepackage[utf8]{inputenc}

% Setup for fullpage use
\usepackage{fullpage}

% Uncomment some of the following if you use the features
%
% Running Headers and footers
%\usepackage{fancyheadings}

% Multipart figures
%\usepackage{subfigure}

% More symbols
\usepackage{amsmath}
\usepackage{amssymb}
%\usepackage{latexsym}

% Surround parts of graphics with box
\usepackage{boxedminipage}

% Package for including code in the document
\usepackage{listings}

% If you want to generate a toc for each chapter (use with book)
\usepackage{minitoc}

% This is now the recommended way for checking for PDFLaTeX:
\usepackage{ifpdf}

%\newif\ifpdf
%\ifx\pdfoutput\undefined
%\pdffalse % we are not running PDFLaTeX
%\else
%\pdfoutput=1 % we are running PDFLaTeX
%\pdftrue
%\fi

\ifpdf
\usepackage[pdftex]{graphicx}
\else
\usepackage{graphicx}
\fi
\title{Notulen - OGO 1.2, groep 2}
\author{ Sander \\ }

\date{14 februari 2007}

\begin{document}

\ifpdf
\DeclareGraphicsExtensions{.pdf, .jpg, .tif}
\else
\DeclareGraphicsExtensions{.eps, .jpg}
\fi

\maketitle


\section{Opening vergadering: 14:25 uur }
  \begin{itemize}
    \item Voorzitter: Etienne
    \item Notulist: Sander
    \item Afwezige leden: Giso, met bericht
    \item De agenda wordt zonder wijzigingen overgenomen
  \end{itemize}


\section{Bespreken van notulen vorige vergadering}
  \begin{itemize}
     \item Tutor heeft de notulen van de vorige vergadering niet
     ontvangen.
     \item Er stonden niet alle afspraken in.
     \item Er komt onder elke notulen een overzicht van afspraken.
  \end{itemize}

\section{Mededelingen Voorzitter}

  \begin{itemize}
     \item De data van de presentatie en de demonstratie staan op de
     website van ogo.
     \item De database moet ingevuld worden, zodat de demonstratie
     verloopt. De verantwoording moet afgesproken worden onder punt
     9: Planning.
     \item Niet alle scenario's worden behandeld tijdens de
     demonstratie.
     \item Voor elk scenario is een iemand verantwoordelijk, geen
     twee.
     \item GUI aan de expert tonen.
     \item 15 maart eindpresentatie in Paviljoen.
     \item Handleiding: is nog niets aan gedaan. (2 maart deadline) Etienne gaat een template maken.
     \item Giso moet de handleiding van het formulier van Xander maken.
     \item SVN: nog niet gelukt, dus we houden de server van Xander aan.
  \end{itemize}

\section{Mededelingen tutor}

    \begin{itemize}
        \item Handleiding: een proefversie op 2 maart heeft geen zin.
        \item De proefdemo is 15 feb op vloer 10.
        \item Donderdag 1 maart is de volgende reguliere vergadering.
        \item Er moeten 3 mogelijke uitbreidingen ingeleverd worden, bij meneer Voorhoeve, as vrijdag. Dit stond in het werkplan als 'vergadering'.
    \end{itemize}

\section{Mededelingen leden}

\begin{itemize}
   \item Sander: de tijden van aanvang worden te los ge\"interpreteerd: als je te laat bent, bel dan, ook als het maar vijf minuten is!
\end{itemize}

\section{Opdracht bespreking}

\begin{itemize}
    \item Na de vergadering bespreken we wat er nog ontbreekt in de GUI.
    \item De beveiliging is nog niet ge\"implementeerd.
    \item De veranderingen moeten nog gemaakt worden. Vanavond moet er bij Sander in zijn mailbox van iedereen twee idee\"en tot verbetering of uitbreiding van de database
\end{itemize}

\section{Verslaglegging}

\begin{itemize}
   \item Productverslag: Giso is er niet en kan geen statusupdate geven. Het moet ongeveer 9 \'a 10 kantjes lang worden.
   \item Procesverslag: iedereen schrijft een stukje over het groepsproces, met blz 4 van de projectwijzer als leidraad en levert dat voor of op 9 maart in bij Sander.
\end{itemize}

\section{Planning}

\begin{itemize}
   \item Volgende werkweek (de week na carnaval): handleiding \& projectverslag \& demonstratie.
\end{itemize}

\section{Actielijst}

\begin{itemize}
    \item Onder alle notulen komt een actielijst
    \item GUI aan expert laten zien deze week
    \item Etienne maakt een template voor de handleiding
    \item Jurjen gaat Giso vragen of hij zijn taken van deze week af kan krijgen
    \item Iedereen (behalve Giso) omschrijft twee verbeteringen aan de database en mailt die voor morgen naar Sander.
    \item Iedereen schrijft een stukje over het groepsproces, deadline 9 maart.
\end{itemize}

\section{Sluiting}
  \begin{itemize}
    \item De voorzitter sluit de vergadering om 15:20, de volgende vergadering is donderdag 1 maart.
  \end{itemize}

$\boxtimes$

\end{document}
