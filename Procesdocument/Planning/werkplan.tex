\documentclass[10pt,twoside]{report}
    %\usepackage{tuestyle}
    \usepackage{amsmath}
    \usepackage{amssymb}
    \usepackage{amsthm}
    \usepackage{array}
    \usepackage{xy}
    \topmargin 0.0cm
    \textheight 7.7in
    \textwidth 6.5in
    \oddsidemargin 0.0cm
    \evensidemargin 0.0cm
    \headheight 0.0cm
    \headsep 0.0cm
    \topskip 1.0cm

\author{Nick van der Veeken}
\title{Deadlines, trainingen en besprekingen}
\begin{document}

{\Large \textbf{Werkplan}} {\Large OGO Groep 2}, versie 1.009\\\\

\begin{enumerate}
\item \textbf{Besprekingen met expert}
    \begin{itemize}
        \item \checkmark \space Doorspreken producten d.d. 8 december met expert (eind week 2)
        \item \checkmark \space Doorspreken producten week 3 en 4 met expert (eind week 3)
    \end{itemize}

\item \textbf{Trainingen}
    \begin{itemize}
        \item \checkmark \space \textbf{Access-training 1:} 13:30 tot 15:15, woensdag 29 november, AUD 10 (week 1)
        \item \checkmark \space \textbf{Introductie Visio:} 1e uur, dinsdag 4 december (week 2)
        \item \checkmark \space \textbf{Vergadertraining 1:} 14:30 tot 15:30, donderdag 7 december, locatie: Traverse 4 (week 2)
        \item \checkmark \space \textbf{Vergadertraining 2:} 14:30 tot 15:30, donderdag 21 december, locatie: Traverse 5 (week 4)
        \item \checkmark \space \textbf{Access-training 2:} 13:30 tot 15:15, woensdag 10 januari, AUD 10 (week 5)
    \end{itemize}

\item \textbf{Studentenoverleg}
    \begin{itemize}
        \item \checkmark \space 13 december, 13:00, HG 5.95, afgevaardigde: Nick
        \item $\chi$ \space 10 januari, 13:00, HG 5.95, afgevaardigde: Jurjen
        \item \checkmark \space 14 februari, 13:00, HG 5.95, afgevaardigde: Etienne
    \end{itemize}

\item \textbf{Deadline- en taakoverzicht}\\
    \\
    Interne deadlines staan op 23:59, de woensdag voor de eigenlijke harde
    deadline. Dit om te zorgen dat er nog nagekeken en eventueel
    verbeterd kan worden.

    \begin{itemize}
      \item 1 december, 12:00 (Week 1)
        \begin{itemize}
          \item \checkmark \space \textbf{Oefencasus 'Klein Ziekenhuis':} Access-database, inleveren via Studyweb
        \end{itemize}
      \item 8 december, 12:00 (week 2)
        \begin{itemize}
          \item \checkmark \space \textbf{Werkplan (Nick):} PDF, op papier bij de tutor en in de groepsfolder op Studyweb
          \item \checkmark \space \textbf{Aangepaste use cases (Jurjen):} PDF, op papier bij expert en in de groepsfolder
          \item \checkmark \space \textbf{Taakoverzicht (Sander):} PDF, op papier bij expert en in de groepsfolder
          \item \checkmark \space \textbf{Lijst van queries en updates (Giso):} PDF, op papier bij expert en in de groepsfolder
        \end{itemize}
      \item 15 december, 12:00 (week 3)
        \begin{itemize}
          \item \checkmark \space \textbf{ER-model, eerste versie (Etienne):} Visio(of gelijk/meerwaardig), op papier bij de expert en in de groepsfolder
        \end{itemize}
      \item 22 december, 12:00 (week 4)
        \begin{itemize}
          \item \checkmark \space \textbf{ER-model, eindversie (Xander):} Visio, op papier bij de expert en in de groepsfolder
          \item \checkmark \space \textbf{Prototype GUI (Sander):} Access-database, per email bij de expert en in de groepsfolder
        \end{itemize}
      \item 12 januari, 12:00 (week 5)
        \begin{itemize}
          \item \checkmark \space \textbf{Globaal datamodel (Nick):} Op papier bij de expert en in de groepsfolder
        \end{itemize}
      \item 8 februari, 14:00, tijdens de vergadering (week 6)
        \begin{itemize}
          \item \checkmark \space \textbf{Gevulde database met werkende formulieren en rapporten (Jurjen):} Aan de tutor laten zien en in de groepsfolder plaatsen
        \end{itemize}
      \item 12 februari, 12:00 (week 7)
        \begin{itemize}
          \item \checkmark \space \textbf{Scenario's voor oefendemonstratie (Nick):} Op papier bij de tutor en in de groepsfolder.
        \end{itemize}
      \item 14 februari, 23:59 (week 7)
         \begin{itemize}
          \item \checkmark \space \textbf{Suggesties uitbreiding database mailen naar Sander (Iedereen):}
          Mail twee suggesties voor toegevoegde functionaliteit aan
          de database naar Sander. Geef naast een omschrijving van
          de functionaliteit (een paar regels) ook nog een suggestie
          voor een gemakkelijke implementatie van de
          functionaliteit.
        \end{itemize}
      \item 15 februari, 14:00 (week 7)
        \begin{itemize}
          \item \checkmark \space \textbf{Voorlopige handleiding demonstratie scenario's (Nick):}
          Een letterlijke omschrijving van wat in welke velden in de
          database moet worden ingevuld om de scenario's te kunnen
          demonstreren.
          \item \checkmark \space \textbf{Oefendemonstratie (Nick):} De tutor krijgt een demonstratie van de scenario's die
          ontworpen zijn aan hand van de use cases.
        \end{itemize}
      \item 16 februari, 12:00 (week 7)
        \begin{itemize}
          \item \checkmark \space \textbf{Suggesties voor toegevoegde functionaliteit aan de database (Sander):} Op papier bij de expert en in de groepsfolder
          \item \checkmark \space \textbf{Scenario's voor oefendemonstratie (Nick):} De scenario's inleveren op papier bij de projectco�rdinator en in de
          groepsfolder.
        \end{itemize}
      \item 28 februari, 23:59 (week 8)
        \begin{itemize}
          \item \checkmark \space \textbf{Voorlopige versie gebruikershandleiding
          (Etienne):} Aan de tutor laten zien ter evaluatie.
        \end{itemize}
      \item 2 maart, 12:00 (week 8)
        \begin{itemize}
            \item \checkmark \space \textbf{Proefversie projectverslag (Giso en
            Sander):} Op papier bij de tutor (ter controle van de
            structuur, nog niet alle tekst hoeft af te zijn, als de
            hoofdstukken en kopjes er maar staan).
            \item \checkmark \space \textbf{Gebruikershandleiding (Etienne):} Op papier bij de projectco�rdinator en in de groepsfolder
        \end{itemize}
      \item 8 maart, 13:45 - 14:00, locatie ?? (week 8)
        \begin{itemize}
          \item \checkmark \space \textbf{Handleiding demonstratiescenario's (Nick):}
          Een letterlijke omschrijving van wat in welke velden in de
          database moet worden ingevuld om de scenario's te kunnen
          demonstreren.
          \item \checkmark \space \textbf{Demonstratie (Nick):} Demonstratie van de database aan de projectco�rdinator en de
          expert
          14:15
        \end{itemize}
      \item 14 maart, 23:59 (week 8)
        \begin{itemize}
          \item \checkmark \space \textbf{Definitieve versie evaluatie OGO-project en reflectie (Iedereen):} Naar Sander mailen
        \end{itemize}
      \item 14 maart, 23:59 (week 9)
        \begin{itemize}
          \item \checkmark \space \textbf{Vragen demonstratie voor andere groepen
            (Nick):} Voor eigen gebruik, aantal generieke vragen verzinnen
            voor de beiden varianten presentaties om aan de andere
            groepen te stellen wanneer we niet on-the-fly vragen
            kunnen bedenken.
        \end{itemize}
      \item 15 maart, 12:45 tot 14:00, PAV M.23 (week 9)
        \begin{itemize}
          \item \checkmark \space \textbf{Presentatie (Etienne):} Klantgerichte
          presentatie, met beamer. Demonstratie van het product met
          bijbehorend verkooppraatje
        \end{itemize}
      \item 16 maart, 12:00 (week 10)
        \begin{itemize}
          \item \checkmark \space \textbf{Applicatiesoftware, incl. gebruikersnamen en wachtwoorden (Giso):} Op cd bij de projectco�rdinator en in de groepsfolder
          \item \checkmark \space \textbf{Logboek (Nick):} Op papier bij de tutor en in de groepsfolder (let op de naamgeving)
          \item \checkmark \space \textbf{Procesdocument (Sander):} Op papier bij de projectco�rdinator en in de groepsfolder
          \item \checkmark \space \textbf{Productdocument (Ontwerpbeslissingen: Giso):} Op papier bij de projectco�rdinator en in de groepsfolder
        \end{itemize}
    \end{itemize}

\item \textbf{Planning}\\
  \\
  Werklast wordt genoemd als aantal personen, beginnen in welke week
  en hoeveel uur per persoon die eraan werkt. Het werk moet af zijn voor de deadline die hierboven voor het werk genoemd staat. Dus
  bijvoorbeeld: ``ER-model: Beginnen in week 2, 3 personen, 8 uur'' houdt
  in: 24 uur werklast, begint in week 2 en moet af zijn in week 4
  (zie deadline ER-model hierboven).\\

  Al af:
  \begin{itemize}
    \item \textbf{Globaal datamodel:} Beginnen: direct na de kerstvakantie. Eventueel het huidige ER-model
    aanpassen, bij elkaar genomen 1 persoon, 2 uur
    \item \textbf{Documentatie database bijwerken:} Zolang als de
    database nog wordt veranderd, 1 persoon, 1 uur per week
    \item \textbf{Gevulde database met werkende formulieren en
    rapporten:} Beginnen: na de kerstvakantie. Werklast: 3 weken, zes personen, 8
    uur per persoon
    \item \textbf{Suggesties toevoegen features aan database:} Beginnen in week 6, tijdens de OGO-bijeenkomsten. Bespreking: gehele groep, 1 uur. Uitwerken in het begin van week
    12, gehele groep, 1 uur. Samenvoegen, 1 persoon, 1 uur
    \item \textbf{Scenario's voor de oefendemonstratie:} Bedenken en op papier uitwerken van de
    scenario's: 2 personen, 4 uur in week 6 of 7. De demonstratie zelf zal niet meer dan 20
    minuten in beslag nemen.
    \item \textbf{Gebruikershandleiding:} Beginnen: als de
    formulieren klaar zijn. Werklast: 2 weken, 3 man, 8 uur per week
    \item \textbf{Proefversie projectverslag:} Inleveren in week 8
    \item \textbf{Applicatiesoftware op cd zetten, incl. gebruikersnamen en
    wachtwoorden:} Beginnen: in de week van de deadline voor de
    software, werklast: 1 persoon, 2 uur
    \item \textbf{Logboek:} Iedere week bijwerken (5 minuten per persoon per week), in week 10 als PDF bij het procesdocument
    voegen, 1 persoon, 30 minuten
    \item \textbf{Procesdocument:} Al gemaakte delen bijeenvoegen in
    week 15, 2 personen, 4 uur. Afwerken in week 10, 2 personen, 8 uur
    \item \textbf{Productdocument:} Gemaakte delen bijeenvoegen in
    week 14 en 15, 2 personen, 4 uur. Afwerken in week 10, 2 personen, 8 uur
    \item \textbf{Presentatie:} Beginnen: 3 weken voor het houden van
    de presentatie; werklast: 2 man, 3 maal 4 uur.\\
  \end{itemize}

\item \textbf{Logboek}\\
    \\
    Het logboek is bij ieder groepslid op dinsdagavond
    23:59 zo ver mogelijk bijgewerkt en op de SVN-server gezet.

\item \textbf{Voorziene afwezigheid}
    \begin{itemize}
        \item \textbf{21 december:} Etienne en Nick, vanaf 15:30 (voor de
        kerstlunch W\&I).
        \item \textbf{vanaf begin februari:} Xander is gestopt met
        zijn studie, dus hij zal niet meer aanwezig zijn bij
        bijeenkomsten en verder ook niet meer meewerken aan het
        project.
        \item \textbf{14 februari:} Giso is ziek, dus komt niet naar
        de OGO-bijeenkomst.
    \end{itemize}

\end{enumerate}

\end{document}
